\documentclass[a4paper, 14pt]{article}
\usepackage[utf8]{inputenc}
%%% Дополнительная работа с математикой
\usepackage{amsmath,amsfonts,amssymb,amsthm,mathtools} % AMS
\usepackage{wrapfig,lipsum,cleveref}
\usepackage{icomma} % "Умная" запятая: $0,2$ --- число, $0, 2$ --- перечисление
\usepackage{listings}
\usepackage{color}
\usepackage{geometry} % Простой способ задавать поля
\usepackage{longtable}

\linespread{1.5}

\geometry{top=25mm}
\geometry{bottom=35mm}
\geometry{left=35mm}
\geometry{right=20mm}

\definecolor{codegreen}{rgb}{0,0.6,0}
\definecolor{codegray}{rgb}{0.5,0.5,0.5}
\definecolor{codepurple}{rgb}{0.58,0,0.82}
\definecolor{backcolour}{rgb}{0.95,0.95,0.92}
\definecolor{codered}{rgb}{0.130,0,0}
\definecolor{codedodgerblue}{rgb}{0.176,0.224,0.230}
\lstdefinestyle{mystyle}{
	backgroundcolor=\color{backcolour},   
	commentstyle=\color{codedodgerblue},
	keywordstyle=\color{codegreen},
	numberstyle=\tiny\color{codegreen},
	stringstyle=\color{codered},
	basicstyle=\footnotesize,
	breakatwhitespace=false,         
	breaklines=true,                 
	captionpos=b,                    
	keepspaces=true,                 
	numbers=left,                    
	numbersep=5pt,                  
	showspaces=false,                
	showstringspaces=false,
	showtabs=false,                  
	tabsize=2
}
\lstset{style=mystyle}

%% Номера формул
\mathtoolsset{showonlyrefs=true} % Показывать номера только у тех формул, на которые есть \eqref{} в тексте.

%% Шрифты
\usepackage{euscript}	 % Шрифт Евклид
\usepackage{mathrsfs} % Красивый матшрифт

%% Свои команды
\DeclareMathOperator{\sgn}{\mathop{sgn}}

%% Перенос знаков в формулах (по Львовскому)
\newcommand*{\hm}[1]{#1\nobreak\discretionary{}
{\hbox{$\mathsurround=0pt #1$}}{}}


\title{Использование LSTM нейронных сетей для прогнозирования макроэкономических рядов.}
\usepackage{cmap}					% поиск в PDF
\usepackage[T2A]{fontenc}			% кодировка
\usepackage[utf8]{inputenc}			% кодировка исходного текста
\usepackage[english,russian]{babel}	% локализация и переносы
\usepackage{graphicx}
\graphicspath{{pictures/}}
\DeclareGraphicsExtensions{.pdf,.png,.jpg}
\author{Гармидер Петр}
\date{\today}
\begin{document}


\thispagestyle{empty}
\begin{center}
	\textbf{ПРАВИТЕЛЬСТВО РОССИЙСКОЙ ФЕДЕРАЦИИ}\\
	\vspace{2ex}
	\textbf{Федеральное государственное автономное\\ образовательное учреждение высшего образования}
	
	\vspace{2ex}
	
	\textbf{Национальный исследовательский университет \\ <<Высшая школа экономики>>}
	
	\vspace{8ex}
	\begin{flushright}
		Факультет экономических наук\\
		Образовательная программа <<Экономика>>
	\end{flushright}
\end{center}
\vspace{9ex}

\begin{center}
	{\textbf{КУРСОВАЯ РАБОТА
	}}
	\vspace{1ex}
	
	На тему <<Вариация алгоритма кросс-валидации со взвешиванием наблюдений>>
\end{center}
\vspace{1ex}
\begin{flushright}
	\noindent
	Студент группы БЭК161\\Гармидер Петр Александрович\\
	\vspace{13ex}
	Научный руководитель:\\
	Борис Демешев
	
\end{flushright}	

\vfill

\begin{center}
	Москва 2019
	
\end{center}
\newpage

\tableofcontents

\newpage




\newpage
\bibliographystyle{utf8gost705u}  %% стилевой файл для оформления по ГОСТу
\bibliography{biblio}     %% имя библиографической базы (bib-файла) 


\end{document}
